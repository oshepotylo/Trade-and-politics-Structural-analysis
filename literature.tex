\section{Literature Review: Trade and politics}

\subsection{Institutions}
The impact of (formal and informal) institutions on trade can be modelled through GDP growth. The mechanism here is reasonably straightforward: good institutions spur economic growth and increase total factor productivity in the long run, which has a positive impact on bilateral trade flows through higher income and expenditure in the affected countries \citep{Acemoglu2001, Acemoglu2002}. On the other hand, the development of stable institutions is also conducive to long distance trade; particularly when we consider products that are complex and technologically advanced, as these rely more heavily on the capacity to write contracts. Therefore, countries with strong and stable institutions have a comparative advantage in goods that require well-developed contracts and their enforcement \citep{Nunn2015, Levchenko2007a}. In other words, the improvement of institutions may be equivalent to reducing tariffs \citep{anderson1999trade}, where the reduction of other regulations and formalities may be equivalent to lowering non-tariff barriers. Additionally, in environments where formal rules work poorly, reputation works as a substitute for contract enforcement \citep{Greif1993ContractCoalition., Greif1994a}. Finally, it has been shown that poor institutions can also impact on the geography of trade \citep{lambsdorff1998empirical}. 

\subsection{Trade and political relations}

Political scientists have long debated the interplay of politics and trade. Trade follows the flagship paradigms of mercantilism in the 18th century and imperialists in the 19th century. Research on the impact of politics on trade has confirmed that politics and bilateral conflicts influence trade \citep{Pollins1989DoesFlag, Pollins1989ConflictFlows, Martin2008, Glick2010}. At the same time, there is an extensive literature regarding trade promoting peace \citep{Polachek1980ConflictTrade,Oneal2001ClearPiece}. Evidently, as the direction of causality may go both ways, it should be modelled accounting for both effects, either via simultaneous equation modelling or using an instrumental variable approach. \cite{Keshk2004} apply a simultaneous equation model and find a uni-directional impact; namely, that politics influences bilateral trade. Similarly, \cite{Glick2010} demonstrate that the endogeneity issue in the gravity model framework can be controlled for by including country-pair effects into a fully specified gravity model.

Therefore, the impact of institutions on trade can be modelled as strengthening bilateral political links, which may facilitate trade with countries that are politically and institutionally aligned. For a long-time World Empires promoted trade between metropolises and colonies; ancient Greeks created cities splintered from the key metropolises of Athens and Sparta; Rome built an infrastructure of roads and unified the laws across its provinces. Similarly, the British Empire encouraged trade with its colonies. Reversely, the process of de-colonization reduced trade between metropolies and their former colonies \citep{head2010erosion}. 


\subsection{Channels of political impact on bilateral trade flows}

\subsubsection*{Demand side}

Evidence suggests that consumers incorporate a desire to express goodwill and solidarity towards politically close nations, while they demonstrate a negative attitude and boycott goods from political adversaries.  \cite{Michaels2010FreedomFries} estimate an 8 \% drop in bilateral trade between the US and France as a response to the French opposition to the Iraq war in 2003. Similarly, \cite{yazigi2014syria} report an increase in Syrian trade with its current allies, Russia and Iran, despite the negative effect of the civil war on its economy; while Syria also experienced a marked drop in exports and imports to/from European countries.

\subsubsection*{Supply-side and trade costs}

Countries sacrifice economic efficiency to reach political goals; for example, Russia provides price discounts on natural gas to buy the loyalty of particular governments.  Trade follows the flag; politically motivated favors often pay off, since more loyal governments give preferential treatment to countries they draw their political support from. Some scholars argue that bilateral trade increases after diplomatic exchange \citep{rose2007foreign} and state visits between countries \citep{nitsch2007state}. Conversely, \cite{Fuchs2013PayingTrade}, find evidence of a 'Dalai Lama Effect' that reduces exports to China for around two years after receiving the Dalai Lama.  

\subsubsection*{Trust and reputation}

More broadly, trust between nations is an important determinant of economic ties. \cite{Guiso2009} found that a standard deviation increase in trust of importer into exporter increases exports by 10 \%. \cite{Disdier2007} use the same data and find that a 5 percentage point improvement in bilateral trust increases imports by 5 \%. 

\subsubsection*{Trade policy uncertainty}

Trade policy uncertainty (TPU) and political risk may reduce trade even when countries are aligned politically. When economic agents consider foreign economic relations, they take into account the probability of different trade policies of the foreign states. There have been a number of recent studies that demonstrate that a reduction in TPU increases trade, especially for products that require sunk cost investments. \cite{Handley2015} show that the expectation of Portugal's accession to into the EEC increased firm export entry and sales even before they became members. \cite{Feng2017} show that China's WTO accession  increased Chinese firm export entries.


