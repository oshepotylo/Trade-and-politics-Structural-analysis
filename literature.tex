\section{Literature: Institutions and trade}
Impact of institutions on trade can be modelled directly as strengthening bilateral political links may facilitate trade with countries that are politically and institutionally aligned. World Empires for long-time in the past promoted trade between metropolises and colonies. Ancient Greeks created poleis splintered from metropolises of Athens or Sparta. Rome built infrastructure of roads and unified laws across its provinces. British Empire encourage trade with its colonies. Recent evidence show that US has stronger trade ties with countries it is politically aligned. This literature says that conditional on national institutions, politically aligned countries trade more.
The impact can also be modelled indirectly as a consequence of improvement in institutions on GDP. The mechanism here is rather straightforward. Good institutions spur economic growth, which has a positive impact on all bilateral trade.
Literature on trade and institutions posits crucial importance of rule of law for products that require contractability, so countries with strong rule of law will have comparative advantage in goods that are more contractable.
Formal institutions also are conducive for long distance trade, as 

