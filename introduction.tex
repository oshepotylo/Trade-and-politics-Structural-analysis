\section{Introduction}

The academic literature discussing the Chinese-led Belt and Road Initiative (BRI; also translated as ‘One Belt One Road’, OBOR) is growing rapidly. However, this literature lacks studies based on rigorous empirical economic modelling. The small number of exceptions include \cite{Herrero2017ChinasGains} and a group of studies emerging from the World Bank; particularly noteworthy is the work of \cite {de2018much}, which estimates the trade cost reductions due to the BRI. Therefore, important gaps remain; this paper fills one such gap by looking at how the changes in political stability and political alignment, derived from the BRI, impact on trade and welfare.        
\\
\indent The BRI involves six corridors, running through numerous countries with medium-high risk of political unrest and/or conflict. However, Chinese policy makers are keen to point out that China has no desire to be involved in military-led peace building. Instead, there has been the emergence of a broad narrative referred to as \textit{The New Type of Great Power Relations}. The central pillars of this model are non-conflict, non-confrontation, mutual respect, co-operation and shared benefit. In other words, Chinese policy makers argue that they are looking to utilize China's economic strength, via initiatives such as the BRI, to develop its discursive power to keep the peace \citep{zhao2016china}.  
 \\
 \indent Therefore, the BRI could deliver improvements in political stability and institutions via two channels. First, economic benefits that support the creation of higher quality institutions and create a demand for their development, which could be broadly summarized as the development channel \citep{rigobon2005rule}. Second, soft intervention to protect the BRI investments and trade routes. The potential for institutional improvements via this second channel can be understood with reference to China's role in Africa’s security architecture, where there have been substantial shifts in the nature of Chinese involvement in the UN Security Council and peace building (e.g. brokering peace in South Sudan) \citep{alden2017china}. These examples may have been calculated trial runs, as Chinese politicians consider a shift away from their non-involvement stance, due to a growing acceptance that economic leverage will need to be accompanied by soft intervention. Therefore, this paper will model the impact of these potential improvements in political stability and institutions. In particular, we investigate the role of political stability, measured by the World Bank Governance indicator of political stability, and political alignment, measured by affinity of nations and ideal points distance indices \citep{Bailey2017EstimatingData} in promoting trade between China and the BRI countries. 
 \\
 \indent To the best of our knowledge this is the first paper that explores the trade and welfare effect of improved political stability and political alignment derived from the the BRI. We use a  structural model of global trade by \cite{Anderson2018GEPPML:PPML}, amended to reflect political preferences in consumption as well as the modification of production process that depend on stable institutions that protect property rights and political stability. We also incorporate political factors into trade costs. The model is used to generate welfare effects of the BRI in conditional and general equilibrium analysis. Our analysis considers the BRI impact across two dimensions: political environment and trade costs; therefore, we construct a number of scenarios that permit us to compare the welfare gains across each of these dimensions. 
\\
\indent Our results indicate that the BRI benefits come from reduced trade costs, reduced bilateral political uncertainty, military alliances, and more political stability in the BRI countries. Under the conditional general equilibrium estimations, when we consider only the impact of direct changes, transport cost reductions resulting from the BRI would boost welfare in China by 1.59\% and global welfare by 0.28\%. Greater political stability would be most beneficial for the Middle East and South Asia regions. More aligned political affinity would benefit Chinese welfare by 0.36\%.  And even closer defense alliances with the BRI countries would have the strongest impact on trade and welfare, both in China, increasing welfare by 3.55\%, and globally, increasing welfare by 0.61\%.
The effects are stronger for the full general equilibrium results, since there is positive feedback from lower trade costs to price changes as well as gains in incomes and expenditures across all countries. The impact of the BRI on China would be 1.81\% increase in welfare, while globally, there would be 0.49\% increase in welfare. More aligned political worldviews would boost welfare in China by 0.45\% and globally by 0.12\%.  Forming defense alliances would be the most beneficial, increasing welfare in China by 4.11\%. Finally, higher political stability would boost welfare in the Middle East by 4.12\% and in South Asia by 8.82\%.
\\
\indent The rest of the paper is structured as follows. Section 2 discusses literature, section 3 presents the model, section 4 introduces counterfactual scenarios, section 5 discusses data, section 6 presents results, and section 7 concludes.
