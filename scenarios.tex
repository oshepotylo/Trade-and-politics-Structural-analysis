\section{Scenarios}
Our analysis considers the BRI impact across two dimensions: political environment and trade costs; therefore, we construct a number of scenarios that permit us to compare the welfare gains across each of these dimensions (see Table \ref{tab:scenarios}).

\subsection{Trade cost reductions}
The basis for our first scenario is transport cost reductions, while we account for the complexity and uncertainty of the arrangements by considering a range of transport cost reductions across China, BRI countries and the EU of 15 percent and estimating the associated welfare impacts. 

\subsection{Political alignment and stabilization}
We consider the effects of closer political alignment, including narrowing the gap between China and the BRI countries in their UN voting (affinity of nations) and in their view on the global liberal order (ideal points difference). As a stronger version of political affinity, we consider forming defense alliances between China and the BRI countries. Finally, we look at a scenario where political stability in the BRI countries increases to the average level of political stability in the European and Central Asian countries in 2014.

\begin{table}
    \caption{Counterfactual scenarios}
    \label{tab:scenarios}
    \centering
    \begin{tabular}{l|l}
    \toprule
        Scenario  & Brief Description  \\
        \midrule
         BRI & \multicolumn{1}{m{11cm}}{Trade cost reduction of 15\% between China and BRI countries} \\
         DEF & \multicolumn{1}{m{11cm}}{Forming defense alliances between China and BRI countries} \\
         PA & \multicolumn{1}{m{11cm}}{Strengthened political affinity (closer ideal points relative to the current global world order, see \cite{Bailey2017EstimatingData})} \\
         PS & \multicolumn{1}{m{11cm}}{Higher political stability in BRI countries (increasing to the level that is average in the European and Central Asian region)} \\
         \bottomrule
    \end{tabular}
    
\end{table}

